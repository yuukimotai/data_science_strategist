\documentclass{article}
\usepackage{luatexja}
\usepackage{amsmath}
\usepackage{amsfonts}
\usepackage{amssymb}
\usepackage{fontspec}
\usepackage{xcolor}

\begin{document}
\textbf{【平方完成】}\\
y = ax^2 + bx + c \textbf{を平方完成する}\\
\text{①aに注目して丸かっこで括る}\\
y = a(x^2 + \frac{b}{a}x) + c\\
\text{②上記のaで括った丸かっこ内を完全平方式に変換する}\\
y = a \left( x^2 + \frac{b}{a}x + \left( \frac{b}{2a} \right)^2 - \left( \frac{b}{2a} \right)^2 \right) + c\\
{③(x + \frac{b}{2a})^2の形に変換して余分な項をだす}\\
y = a \left( \left( x + \frac{b}{2a} \right)^2 - \left( \frac{b}{2a} \right)^2 \right) + c\\
\text{④最終形}\\
y = a \left( x + \frac{b}{2a} \right)^2 - \frac{b^2}{4a} + c\\
y = a \left( x + \frac{b}{2a} \right)^2 + \left( c - \frac{b^2}{4a} \right)\\
\text{メモ}\\
\textbf{符号が間違っているので注意 https://www.meikogijuku.jp/meiko-plus/study/completing-the-square.html}
\end{document}