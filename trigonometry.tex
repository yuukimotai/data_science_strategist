\documentclass{article}
\usepackage{luatexja}
\usepackage{amsmath}
\usepackage{amsfonts}
\usepackage{amssymb}
\usepackage{fontspec}
\usepackage{xcolor}

\begin{document}
\textbf{【三角関数】}\\
\textbf{扇形の弧の長さと面積}
\begin{itemize}
    \item \(l = \textcolor[rgb]{1,0.5,0.5}{r\theta}\)
    \item \(S = \textcolor[rgb]{1,0.5,0.5}{\frac{1}{2}r^2\theta = \frac{1}{2}rl}\)
\end{itemize}
\textbf{弧度法と度数法の換算式}
\begin{itemize}
    \item \pi = \textcolor[rgb]{1,0.5,0.5}{\( 180^\circ \)}
    \item  1 = \textcolor[rgb]{1,0.5,0.5}{\((\frac{180}{\pi})^\circ\)}
    \item 1^\circ = \textcolor[rgb]{1,0.5,0.5}{\(\frac{\pi}{180}\)}
\end{itemize}
\textbf{三角関数の値}
\begin{itemize}
    \item sin\theta = \textcolor[rgb]{1,0.5,0.5}{\(\frac{y}{r}\)}
    \item cos\theta = \textcolor[rgb]{1,0.5,0.5}{\(\frac{x}{r}\)} 
    \item tan\theta = \textcolor[rgb]{1,0.5,0.5}{\(\frac{y}{x}\)}
\end{itemize}
\textbf{三角関数の相互関係の公式}
\begin{itemize}
    \item tan\theta = \textcolor[rgb]{1,0.5,0.5}{\(\frac{sin\theta}{cos\theta}\)}
    \item 1 = \textcolor[rgb]{1,0.5,0.5}{\(sin^2\theta+cos^2\theta\)}
    \item \textcolor[rgb]{1,0.5,0.5}{\(1+tan^2\theta=\frac{1}{cos^2\theta}\)}
\end{itemize}
\textbf{余割公式}
\end{document}